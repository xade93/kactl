\chapter{Graph}

\section{Fundamentals}
	% \kactlimport{BellmanFord.h}
	\kactlimport{FloydWarshall.h}
	\kactlimport{DifferenceConstraints.h}

\section{Network flow}
	% \kactlimport{PushRelabel.h}
	\kactlimport{MinCostMaxFlow.h}
	\kactlimport{NegativeLoop.h}
	\kactlimport{Dinic.h}
	\kactlimport{BoundedFlow.h}
	\kactlimport{MinCut.h}
	\kactlimport{MaximumWeightClosedSubgraph.h}
	\kactlimport{GlobalMinCut.h}
	\kactlimport{GomoryHu.h}

\section{Matching}
	\kactlimport{DFSMatching.h}
	\kactlimport{MinimumVertexCover.h}
	\kactlimport{WeightedMatching.h}
	\kactlimport{GeneralMatching.h}
	
\section{DFS algorithms}
	\kactlimport{EulerWalk.h}
	\kactlimport{SCC.h}
	\kactlimport{BCC.h}
	\subsection{2 SAT}
	\begin{align*}
		p_i: \neg p_i \to p_i \qquad \neg p_i : p_i \to \neg p_i \\
		p_i \lor p_j: \neg p_i \to p_j, \neg p_j \to p_i \qquad p_i \lor \neg p_j: \neg p_i \to \neg p_j, p_j \to p_i \\
		\neg p_i \lor \neg p_j: p_i \to \neg p_j, p_j \to \neg p_i
	\end{align*}
	If $p_i$ and $\neg p_i$ strongly connected, 2-SAT has no solution. The solution must be the one which has greater topological order (smaller SCC order) (as true point is $p_i$, and wrong point is $p_i + n$)
\section{Coloring}
	\kactlimport{EdgeColoring.h}

\section{Heuristics}
	\kactlimport{MaximalCliques.h}
	\kactlimport{MaximumClique.h}
	\kactlimport{MaximumIndependentSet.h}
	\kactlimport{Triangle.h}

\section{Trees}
	\kactlimport{BinaryLifting.h}
	\kactlimport{HLD.h}
	\kactlimport{CompressTree.h}
	\kactlimport{LinkCutTree.h}
	\kactlimport{DirectedMST.h}
	\kactlimport{TreeHashing.h}
	\kactlimport{Kruskal.h}
	\kactlimport{Dominator.h}
	\kactlimport{CentroidDecomposition.h}

\section{PlannerGraph}
A planar graph satisfy $|V| - |E| + |F| = 2$, in which $|F|$ is the number of faces. 
When $|V| \geq 3$, we have $|E| \leq 3|V|-6$ and $|F|\leq 2|V|-4$. \\

\subsection{DualGraph}
For each surface $F$, select one $v$ inside. 
For all common edge $e$ between $F_0, F_x$, there's an edge $e_{0,x}$ between $v_0, v_x$ and intersect with $e$ at one point. 
Iff $e$ is only one boundary of the face $F_0$, $v_0$ has a self loop and $e_0$ intersect with $e$.  
The weight of $e_0$ is same as $e$.
The Minimum Cut of a Planar graph equals to the shortest distance on its Dual graph.
We can connect $s$ and $t$ to construct a new face as $s'$, and the inf face as $t'$. 
The shortest distance from $s'$ to $d'$ is the answer. 
Note: one edge in dual graph is a cut in origin graph.

\section{Math}
	\subsection{Number of Spanning Trees}
		% I.e. matrix-tree theorem.
		% Source: https://en.wikipedia.org/wiki/Kirchhoff%27s_theorem
		% Test: stress-tests/graph/matrix-tree.cpp
		Create an $N\times N$ matrix \texttt{mat}, and for each edge $a \rightarrow b \in G$, do
		\texttt{mat[a][b]--, mat[b][b]++} (and \texttt{mat[b][a]--, mat[a][a]++} if $G$ is undirected).
		Remove the $i$th row and column and take the determinant; this yields the number of directed spanning trees rooted at $i$
		(if $G$ is undirected, remove any row/column).

	\subsection{Erdős–Gallai theorem}
		% Source: https://en.wikipedia.org/wiki/Erd%C5%91s%E2%80%93Gallai_theorem
		% Test: stress-tests/graph/erdos-gallai.cpp
		A simple graph with node degrees $d_1 \ge \dots \ge d_n$ exists iff $d_1 + \dots + d_n$ is even and for every $k = 1\dots n$,
		\[ \sum _{i=1}^{k}d_{i}\leq k(k-1)+\sum _{i=k+1}^{n}\min(d_{i},k). \]
